\section{Técnicas}
\metroset{block=fill}

\begin{frame}{Geração de Relevo}
    \begin{itemize}[<+- | alert@+>]
        \item \alert<6>{Ruído de Perlin}
        \item Divisões estocásticas
        \item Falhas geológicas
        \item Deposição de sedimentos
        \item Disposição do ponto médio
    \end{itemize}
\end{frame}

\begin{frame}{Noise}
    \begin{block}{Definição}
        \begin{itemize}
            \item Ponto: $Point \in \{ \mathbb{R}^3 \vee \mathbb{R}^2 \vee \mathbb{R}\} $
            \item $ -1 \leq noise(Point) \leq 1$
        \end{itemize}
    \end{block}
\end{frame}

\begin{frame}{Noise}
    \begin{figure}
		\centering
        \includegraphics[width=.7\textwidth]{img/explain/noiseRandom.png}
        \caption{\alert{Noise vs Random}, por \cite{shiffman2012nature}.}
    \end{figure}
\end{frame}

\begin{frame}{Noise}
    \begin{figure}
		\centering
        \includegraphics[width=.7\textwidth]{img/explain/1d2dnoise.png}
        \caption{\alert{1d and 2d Noise}, por \cite{shiffman2012nature}.}
    \end{figure}
    \begin{itemize}
        \item Complexidade para $n$ dimensões: $\mathcal{O}(2^n)$
    \end{itemize}
\end{frame}


\begin{frame}{Ruído de Perlin}
    \begin{itemize}
        \item Quantidade de oitavas: $\theta \in \mathbb{N}$
    \end{itemize}
    
    \begin{block}{Definição}
        $$perlinNoise(Point, \theta) = \sum_{t=0}^{t=\theta} \frac{Noise(Point \cdot 2^{t})}{2^{t}}$$
    \end{block}
\end{frame}


\begin{frame}{Ruído de Perlin}
    \begin{figure}
		\centering
        \includegraphics[width=.7\textwidth]{img/explain/perlin1d.png}
        \caption{Ruído de Perlin 1D, por \cite{elias2000perlin}.}
    \end{figure}
\end{frame}

\begin{frame}{Ruído de Perlin}
    \begin{figure}
		\centering
        \includegraphics[width=.7\textwidth]{img/explain/octaves1.png}
        \caption{$\theta = 1$.}
    \end{figure}
\end{frame}

\begin{frame}{Ruído de Perlin}
    \begin{figure}
		\centering
        \includegraphics[width=.7\textwidth]{img/explain/octaves4.png}
        \caption{$\theta = 4$.}
    \end{figure}
\end{frame}

\begin{frame}{Ruído de Perlin}
    \begin{figure}
		\centering
        \includegraphics[width=.7\textwidth]{img/explain/octaves16.png}
        \caption{$\theta = 16$.}
    \end{figure}
\end{frame}

\begin{frame}{Ruído de Perlin}
    \begin{block}{Caso de borda}
        $$\sum_{t=0}^{t=\theta} \frac{Noise(Point \cdot 2^{t})}{2^{t}}$$
        $$Noise(Point) = 1$$
        $$\sum_{t=0}^{t=\theta} \frac{1}{2^{t}} = 2^{-\theta} (2^{\theta +1}-1)$$
    \end{block}
\end{frame}


\begin{frame}{Ruído de Perlin}
    \begin{equation*}
        \begin{split}
            \visible<+->{max & = \lim_{\theta\to \infty} 2^{-\theta} (2^{\theta +1}-1) \\}
            \visible<+->{& = 2}
        \end{split}
    \end{equation*}
\end{frame}


\begin{frame}{Ruído de Perlin}
    \begin{figure}
		\centering
        \includegraphics[width=.7\textwidth]{img/explain/farLands.jpg}
        \caption{Limitações}
    \end{figure}
\end{frame}

% Mostrar o que é tesselation

